These are the high-\/level design rules which guide the development of Bootstrap's plugin apis. 



\subsection*{D\-A\-T\-A-\/\-A\-T\-T\-R\-I\-B\-U\-T\-E A\-P\-I}

We believe you should be able to use all plugins provided by Bootstrap purely through the markup A\-P\-I without writing a single line of javascript.

We acknowledge that this isn't always the most performant and sometimes it may be desirable to turn this functionality off altogether. Therefore, as of 2.\-0 we provide the ability to disable the data attribute A\-P\-I by unbinding all events on the body namespaced with {\ttfamily 'data-\/api'}. This looks like this\-: \begin{DoxyVerb}$('body').off('.data-api')
\end{DoxyVerb}


To target a specific plugin, just include the plugins name as a namespace along with the data-\/api namespace like this\-: \begin{DoxyVerb}$('body').off('.alert.data-api')
\end{DoxyVerb}






\subsection*{P\-R\-O\-G\-R\-A\-M\-A\-T\-I\-C A\-P\-I}

We also believe you should be able to use all plugins provided by Bootstrap purely through the J\-S A\-P\-I.

All public A\-P\-Is should be single, chainable methods, and return the collection acted upon. \begin{DoxyVerb}$(".btn.danger").button("toggle").addClass("fat")
\end{DoxyVerb}


All methods should accept an optional options object, a string which targets a particular method, or null which initiates the default behavior\-: \begin{DoxyVerb}$("#myModal").modal() // initialized with defaults
$("#myModal").modal({ keyboard: false }) // initialized with now keyboard
$("#myModal").modal('show') // initializes and invokes show immediately afterqwe2
\end{DoxyVerb}






\subsection*{O\-P\-T\-I\-O\-N\-S}

Options should be sparse and add universal value. We should pick the right defaults.

All plugins should have a default object which can be modified to effect all instance's default options. The defaults object should be available via {\ttfamily \$.fn.\-plugin.\-defaults}. \begin{DoxyVerb}$.fn.modal.defaults = { … }
\end{DoxyVerb}


An options definition should take the following form\-: \begin{DoxyVerb}*noun*: *adjective* - describes or modifies a quality of an instance
\end{DoxyVerb}


examples\-: \begin{DoxyVerb}backdrop: true
keyboard: false
placement: 'top'
\end{DoxyVerb}






\subsection*{E\-V\-E\-N\-T\-S}

All events should have an infinitive and past participle form. The infinitive is fired just before an action takes place, the past participle on completion of the action. \begin{DoxyVerb}show | shown
hide | hidden
\end{DoxyVerb}






\subsection*{C\-O\-N\-S\-T\-R\-U\-C\-T\-O\-R\-S}

Each plugin should expose it's raw constructor on a {\ttfamily Constructor} property -- accessed in the following way\-: \begin{DoxyVerb}$.fn.popover.Constructor
\end{DoxyVerb}






\subsection*{D\-A\-T\-A A\-C\-C\-E\-S\-S\-O\-R}

Each plugin stores a copy of the invoked class on an object. This class instance can be accessed directly through j\-Query's data A\-P\-I like this\-: \begin{DoxyVerb}$('[rel=popover]').data('popover') instanceof $.fn.popover.Constructor
\end{DoxyVerb}






\subsection*{D\-A\-T\-A A\-T\-T\-R\-I\-B\-U\-T\-E\-S}

Data attributes should take the following form\-:


\begin{DoxyItemize}
\item data-\/\{\{verb\}\}=\{\{plugin\}\} -\/ defines main interaction
\item data-\/target $|$$|$ href$^\wedge$=\# -\/ defined on \char`\"{}control\char`\"{} element (if element controls an element other than self)
\item data-\/\{\{noun\}\} -\/ defines class instance options
\end{DoxyItemize}

examples\-: \begin{DoxyVerb}// control other targets
data-toggle="modal" data-target="#foo"
data-toggle="collapse" data-target="#foo" data-parent="#bar"

// defined on element they control
data-spy="scroll"

data-dismiss="modal"
data-dismiss="alert"

data-toggle="dropdown"

data-toggle="button"
data-toggle="buttons-checkbox"
data-toggle="buttons-radio"\end{DoxyVerb}


\section*{Installation}

Installation of Family Tree is pretty simple. You just have to follow the following steps\-: \begin{DoxyVerb}1) Clone the repository to your Apache Document Root for eg. in my case (/var/www)

    Execute the following command to clone the repository

        $ git clone https://github.com/piyushparkash/FamilyTree.git

2) Now open the browser and open the Directory

    After the cloning repository A folder named FamilyTree will be created in your document root

    with path that will look something like this (/var/www/FamilyTree)

    Then you can write the following in browser http://localhost/FamilyTree/

3) Make sure you have given proper write permissions to the FamilyTree Folder

    If not then you can do it by executing the following command

        $ chmod -R 777 /var/www/FamilyTree
\end{DoxyVerb}


\section*{A\-U\-T\-H\-O\-R\-S\-:}

Piyush Parkash

Website\-: piyushparkash.\-blogspot.\-com

Email\-: \href{mailto:achyutapiyush@gmail.com}{\tt achyutapiyush@gmail.\-com} 